\documentclass[UTF8]{ctexart}

% ---------------- Packages ---------------------------
\usepackage[usenames,dvipsnames]{color} % for adding color
\usepackage{amsmath, amsthm, amssymb}   % for maths
\usepackage[in,myheadings]{fullpage}    % for setting margins
\usepackage{enumerate}  % for setting the enumerate bullet style
\usepackage{fancyhdr}   % for setting page headers and footers
\usepackage{listings}   % for typesetting code
\usepackage{graphicx}   % for 'enhanced' graphics package
\usepackage{framed}     % for frame borders
\usepackage{float}      % for absolute figure placement (H)
\usepackage{url}        % for typesetting hyperlinks
\usepackage{bm}         % for bold greek letters

%for font
%\usepackage[T1]{fontenc}
%\usepackage{mathptmx}
% ---------------- Environments -----------------------
\newenvironment{pitemize}{
	\vspace{-3mm}
	\begin{itemize}
		\setlength{\itemsep}{1pt}
		\setlength{\parskip}{0pt}
		\setlength{\parsep}{0pt}
	}{\end{itemize}}

\newenvironment{penumerate}{
	\begin{enumerate}
		\vspace{-3mm}
		\setlength{\itemsep}{1pt}
		\setlength{\parskip}{-2pt}
		\setlength{\parsep}{-5pt}
	}{\end{enumerate}}

\newenvironment{definition}[1][Definition]{
	\begin{trivlist}
		\item[\hskip \labelsep {\bfseries #1}]
	}{\end{trivlist}}

\newenvironment{example}[1][Example]{
	\begin{trivlist}
		\item[\hskip \labelsep {\bfseries #1}]
	}{\end{trivlist}}

\newenvironment{summary}[1][Summary]{
	\begin{trivlist}
		\item[\hskip \labelsep {\bfseries #1}]
	}{\end{trivlist}}

% ---------------- General Macros --------------------
\newcommand{\ul}[1]{\underline{#1}}             % underline
\newcommand{\todo}[1]{\textcolor{red}{\textbf{#1}}}  % todo
\newcommand{\qpart}[1]{\subsection{#1}\vspace{-3mm}}
\newcommand{\qsection}[1]{\section{#1}\vspace{-3mm}}

% ---------------- Probability Macros -----------------
\renewcommand{\L}[1]{\mathcal{L}}    % likelihood
\newcommand{\E}[1]{\mathbb{E}}     % expectation
\newcommand{\Cov}[1]{\text{Cov}}   % covariance
\newcommand{\Var}[1]{\text{Var}}   % variance
\newcommand{\where}[1]{\text{ where }}
\newcommand{\subto}[1]{\text{ subject to }}

% ---------------- Linear Algebra Macros --------------
\newcommand{\bs}{\boldsymbol}      % bold greek letters
\newcommand{\norm}[1]{\left\lVert#1\right\rVert} % norm

% ---------------- Number Theory Macros ---------------
\newcommand{\R}{\mathbb{R}}
\newcommand{\N}{\mathcal{N}}

% ---------------- Document Settings ------------------
\setlength{\headheight}{20pt}
\setlength{\textheight}{8in}
\setlength{\footskip}{0.8in}
\setlength{\parskip}{6pt plus 2pt minus 1pt}
\setlength{\parindent}{0pt}
\pagestyle{fancy}

% Headers
\lhead{\\EM算法}
\rhead{\\AHUwlkx}
\begin{document}
\section{极大似然估计ML}
极大似然估计(ML)是用来估计统计模型的参数。
\begin{pitemize}
	\item x :观测数据;
	\item $\theta$ :模型参数;
    \item $p(x;\theta)$:作为参数$\theta$的函数,称作似然函数。
\end{pitemize}
极大似然估计是说既然数据以及存在,那么模型的参数要使得数据的概率$p(x;\theta)$最大:
\begin{align}
% \nonumber % Remove numbering (before each equation)
  \hat{\theta}_{ML} = \arg\max\limits_{\theta} p(x;\theta) \tag{1}
\end{align}

\section{期望最大化EM}
\textbf{EM}算法是用来求解含有隐变量的统计模型的一种方法。

\begin{pitemize}
	\item x :观测数据
	\item $\theta$ :模型参数
	\item z :隐变量
\end{pitemize}
\subsection{基于Jensen不等式的推导}

\begin{align*}
    \log p(x;\theta) &= \log\left[\sum_{z}p(x,z;\theta)\right] \\
    &= \log\left[\sum_{z}q(z)\frac{p(x,z;\theta)}{q(z)}\right] \\
    &\geq \sum_{z}\left[q(z)\log\frac{p(x,z;\theta)}{q(z)}\right] \tag{2}
\end{align*}
当$q(z)=p(z|x;\theta)$时,等式成立。

EM算法有E-step和M-step。E-step是给定参数$\theta^{t}$,求隐变量的分布$q(z)=p(z|x;\theta^{t})$;M-step是将求出来的隐变量的分布$q(z)=p(z|x;\theta^{t})$代入公式(2)的右边得到$\log p(x;\theta)$的一个下界函数:
\begin{align*}
    g_{t}(\theta) &= \sum_{z}\left[p(z|x;\theta^{t})\log\frac{p(x,z;\theta)}{p(z|x;\theta^{t})}\right] \\
    &= \sum_{z}\left[p(z|x;\theta^t)\log p(x, z;\theta)\right] - \sum_{z}\left[p(z|x;\theta^t)\log p(z|x;\theta^t)\right] \\
    &= \mathcal{Q}(\theta|\theta^t) + \mathcal{H}(q) \\
    &= \mathcal{Q}(\theta|\theta^t) + constant \tag{3}
\end{align*}
其中
\begin{align}
  \mathcal{Q}(\theta|\theta^t)=\sum_{z}\left[p(z|x;\theta^t)\log p(x, z;\theta)\right] \tag{4}
\end{align}
以及$\mathcal{H}(q)= \sum_{z}\left[p(z|x;\theta^t)\log p(z|x;\theta^t)\right]$。

总结如下:

\textbf{E-step}:给定$\theta^{t}$, 根据公式(2)求出隐变量的分布:
\begin{align}
    q^{t+1}(z)= p(z|x;\theta^t) \tag{5}
\end{align}

\textbf{M-step}:根据E-step的隐变量分布$q^{t+1}(z)= p(z|x;\theta^t)$,最大化$g_t(\theta)$来求解$\theta$,公式(3)说明了最大化$g_t(\theta)$只需要最大化公式(4)的$\mathcal{Q}(\theta|\theta^t)$:
\begin{align}
    \theta^{t+1} = \arg\max\limits_{\theta}\mathcal{Q}(\theta|\theta^t) \tag{6}
\end{align}


\subsection{基于变分推断框架的推导}
\subsubsection{一个等式}

从如下等式开始:
\begin{align}
  p(x;\theta) = \frac{p(x, z;\theta)}{p(z|x;\theta)} \tag{7}
\end{align}
对(7)式右边的分式的分子与分母同除以$q(z)$后取对数:
\begin{align}
  \log p(x;\theta) = \log\frac{p(x, z;\theta)}{q(z)} - \log\frac{p(z|x;\theta)}{q(z)} \tag{8}
\end{align}
在(8)式两边对$q(z)$求期望:
\begin{align}
  E_{q}\log p(x;\theta) = E_{q}\log\frac{p(x, z;\theta)}{q(z)} - E_{q}\log\frac{p(z|x;\theta)}{q(z)} \tag{9}
\end{align}
由于$E_{q}\log p(x;\theta) = \log p(x;\theta)$,得下式:
\begin{align}
  \textcolor[rgb]{0.00,0.50,0.75}{\log p(x;\theta) = \mathcal{F}(q,\theta)+\mathcal{KL}(q||p) \tag{10}}
\end{align}
其中
\begin{align}
    \mathcal{F}(q,\theta)= E_{q}\log\frac{p(x, z;\theta)}{q(z)} = \sum_{z}\left[q(z)\log\frac{p(x, z;\theta)}{q(z)}\right] \tag{11}
\end{align}
以及
\begin{align}
    \mathcal{KL}(q||p) = - E_{q}\log\frac{p(z|x;\theta)}{q(z)} = \sum_{z}\left[q(z)\log\frac{q(z)}{p(z|x;\theta)}\right] \tag{12}
\end{align}

\subsubsection{E-step和M-step}

\textbf{E-step}:给定$\theta^{t}$, 我们寻找$q(z)$使得\textbf{证据下界}$\mathcal{F}(q,\theta)$最大:
\begin{align*}
    q^{t+1}(z) &= \arg\max\limits_{q}\mathcal{F}(q,\theta^t) \\
    &=\arg\min\limits_{q}\mathcal{KL}(q||p(z|x;\theta^t)  \\
    &= p(z|x;\theta^t) \tag{13}
\end{align*}

Remark:$\mathcal{F}(q,\theta)$又被称为\textbf{变分自由能}。

\textbf{M-step}:我们寻找模型参数$\theta$使得$F(q^{t+1}(z),\theta)$最大:
\begin{align*}
    \theta^{t+1} &= \arg\max\limits_{\theta}\mathcal{F}(q^{t+1}(z),\theta) \\
    &= \arg\max\limits_{\theta} E_{q^{t+1}(z)}\log p(x,z;\theta) \tag{14}
\end{align*}

Remark:$\mathcal{Q}(\theta|\theta^t) = E_{q^{t+1}(z)}\log p(x,z;\theta)$
\end{document} 